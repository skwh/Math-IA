\documentclass{article}
\usepackage{amsthm}
\begin{document}
  \title{Using Circles to Approximate Riemann Sums}
  \author{Evan Derby}
  \date{11 December 2015}
  \maketitle

  \section{Introduction}
    Riemann Sums are a method of approximating the area underneath a curve by
    adding up the areas of shapes, usually rectangles, whose height is bound by
    the function. As the width of these shapes approaches zero, the
    approximation approaches the Riemann integral, the true area underneath the
    curve.

    \[ \displaystyle\sum f(x) \Delta x = \int\limits_a^b f(x)dx \]


    Historically, Riemann's definition was the first rigourous definition of the
    integral of a function on an interval.


\end{document}
